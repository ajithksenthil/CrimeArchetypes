\documentclass[11pt,a4paper]{article}

% Packages
\usepackage[utf8]{inputenc}
\usepackage[T1]{fontenc}
\usepackage{amsmath,amssymb,amsthm}
\usepackage{graphicx}
\usepackage{booktabs}
\usepackage{natbib}
\usepackage{hyperref}
\usepackage{geometry}
\usepackage{caption}
\usepackage{subcaption}
\usepackage{float}
\usepackage{enumitem}
\usepackage{xcolor}

\geometry{margin=1in}

% Custom environments
\newtheorem{definition}{Definition}
\newtheorem{proposition}{Proposition}

% Key insight environment - simple framed box for journal compatibility
\newenvironment{keyinsight}{\begin{quote}\itshape\textbf{Key Insight:} }{\end{quote}}

% Title
\title{Criminal Behavioral Trajectories as Dynamical Systems:\\
A Computational Psychodynamics Approach to\\Pattern Discovery and Intervention Optimization}

\author{
Kristina L. Howell\textsuperscript{1,*} \quad Ajith K. Senthil\textsuperscript{2}\\[12pt]
{\normalsize \textsuperscript{1}Attachment and Close-Relationships Lab, University of Illinois at Urbana-Champaign}\\[4pt]
{\normalsize \textsuperscript{2}Social Affective Neuroscience of Decision Making Lab, University of Southern California}\\[8pt]
{\normalsize \textsuperscript{*}Corresponding author: \texttt{khowell@illinois.edu}}
}

\date{}

\begin{document}

\maketitle

\begin{abstract}
Criminal typologies provide limited guidance on when and how to intervene. We apply Computational Psychodynamics---a framework grounded in the Free Energy Principle---to criminal behavioral trajectory analysis. Life events are modeled as transitions through four motivational states (\textit{Seeking}, \textit{Directing}, \textit{Conferring}, \textit{Revising}), transforming retrospective case data into dynamical fingerprints.

We analyzed 1,246 life events from 26 serial offenders. Transfer entropy analysis revealed \textit{archetypal reincarnation}---predictive relationships between individuals' life trajectories indicating shared generative structure across lifetimes. Three network roles emerged: Sources (archetypal templates whose patterns predict others), Sinks (composite cases predicted by multiple archetypes), and Hubs (bridging connectors). Hierarchical classification identified two primary types---COMPLEX (11.5\%) and FOCUSED (88.5\%)---with seven subtypes.

Using structural causal modeling and do-calculus, we identified the Seeking$\rightarrow$Directing transition (``fantasy-to-action'') as a candidate intervention target. Illustrative counterfactual simulations, based on assumed effect sizes from the forensic treatment literature, suggest early intervention \textit{might} reduce harmful outcomes by 23--47\%, pending empirical validation. State space validation confirmed the four-state representation captures significant temporal structure ($p < 0.0001$), with Markov prediction improving accuracy by 12.6\% over marginal baselines.

This work demonstrates that Computational Psychodynamics generalizes from conversational dynamics to longitudinal criminal career analysis, providing both theoretical integration with life-course criminology and practical decision support for intervention planning.

\vspace{10pt}
\noindent\textbf{Keywords:} computational psychodynamics, criminal trajectories, transfer entropy, Markov chains, active inference, intervention optimization
\end{abstract}

\newpage

%==============================================================================
\section{Introduction}
%==============================================================================

\subsection{The Gap Between Pattern Recognition and Intervention}

Criminal typologies have captivated researchers and practitioners for over a century. From Lombroso's ``born criminal'' to the FBI's organized/disorganized dichotomy \citep{douglas1986criminal}, categorical thinking has shaped how we conceptualize, investigate, and respond to serious violent crime. Yet a fundamental gap persists: \textit{risk assessment tools identify who is dangerous, but provide limited guidance on when and how to intervene}.

Current approaches face three interrelated limitations:
\begin{enumerate}[leftmargin=*]
    \item \textbf{Static classification}: Typologies assign offenders to fixed categories, obscuring developmental trajectories and within-type variation \citep{canter2004organized}.
    \item \textbf{Categorical boundaries}: Forcing continuous behavioral variation into discrete boxes sacrifices information and impedes nuanced understanding.
    \item \textbf{Description without prescription}: Knowing an offender is ``high risk'' provides no actionable guidance on intervention timing or modality.
\end{enumerate}

\subsection{Computational Psychodynamics: A Principled Framework}

We address these limitations by applying \textbf{Computational Psychodynamics} \citep{senthil2026comppsy}, a framework that models behavioral dynamics using principles from Active Inference and the Free Energy Principle \citep{friston2010free, parr2022active}. (Note: The Computational Psychodynamics framework paper is currently under review; the full specification is available from the authors upon request.) The framework provides:

\begin{enumerate}[leftmargin=*]
    \item A \textbf{four-state motivational space} (Seeking, Directing, Conferring, Revising) derived from crossing Self/Other orientation with Explore/Exploit behavioral mode
    \item \textbf{Markov chain modeling} of transitions between states, yielding interpretable dynamical fingerprints
    \item \textbf{Transfer entropy} for quantifying directed influence and detecting shared generative structure across individuals
    \item \textbf{Mathematical grounding} in the Steiner system $S(3,4,8)$, ensuring mutual exclusivity, collective exhaustiveness, and minimal category count
\end{enumerate}

Originally validated on conversational dynamics in meeting settings \citep{senthil2026comppsy}, we demonstrate here that Computational Psychodynamics generalizes to a radically different domain: longitudinal criminal life histories spanning decades.

\begin{keyinsight}
The same four motivational states that capture moment-to-moment conversational dynamics also capture the strategic rhythm of criminal careers---from fantasy development through surveillance to violent action and ritualization.
\end{keyinsight}

\subsection{From Description to Prescription}

Beyond applying Computational Psychodynamics for pattern discovery, we extend the framework to enable \textit{intervention reasoning}. Using structural causal modeling and do-calculus, we develop:

\begin{itemize}[leftmargin=*]
    \item \textbf{Critical transition analysis}: Identifying the Seeking$\rightarrow$Directing transition as the primary ``fantasy-to-action'' pathway
    \item \textbf{Tipping point detection}: Locating points where trajectory reversal becomes increasingly unlikely
    \item \textbf{Counterfactual simulation}: Answering ``What if we had intervened at event $k$?''
    \item \textbf{Intervention optimization}: Selecting protocols and timing for maximum expected harm reduction
\end{itemize}

\subsection{Research Questions}

\begin{enumerate}
    \item Can Computational Psychodynamics reliably classify criminal life events into the four-state motivational space?
    \item What trajectory patterns emerge from Markov analysis of criminal behavioral sequences?
    \item Do archetypal structures exist across unconnected individuals, detectable via transfer entropy?
    \item Can we identify meaningful hierarchical types that balance empirical validity with clinical utility?
    \item How can causal modeling enable principled intervention reasoning?
    \item When and how should we intervene for maximum harm reduction?
    \item How does the theory-driven state space compare to data-driven alternatives in terms of information retention and predictive validity?
\end{enumerate}

%==============================================================================
\section{Theoretical Framework: Computational Psychodynamics}
%==============================================================================

This section summarizes the Computational Psychodynamics framework \citep{senthil2026comppsy} and its adaptation to criminal trajectory analysis.

\subsection{The Free Energy Principle and Active Inference}

The Free Energy Principle posits that adaptive agents minimize expected free energy---a quantity combining \textit{pragmatic value} (achieving preferred outcomes) and \textit{epistemic value} (reducing uncertainty) \citep{friston2010free, parr2022active}. Expected free energy for action $a$ decomposes as:

\begin{equation}
    G(a) = \underbrace{-\mathbb{E}_{Q}[\log P(o \mid a)]}_{\text{pragmatic: expected cost}} - \underbrace{\mathbb{E}_{Q}[\mathcal{H}[P(s \mid o, a)]]}_{\text{epistemic: expected ambiguity}}
\end{equation}

where $Q$ is the agent's posterior beliefs, $P(o \mid a)$ encodes outcome preferences, and the entropy term $\mathcal{H}$ captures uncertainty about hidden states $s$ given observations $o$. Agents select actions that minimize $G$, balancing goal-achievement with information-seeking.

\subsection{The Four Motivational Quadra}

Computational Psychodynamics decomposes the single free-energy objective into four irreducible motivational targets by crossing two dimensions:

\begin{itemize}
    \item \textbf{Self vs. Other}: Whether optimization is for the agent's own model or another's
    \item \textbf{Explore vs. Exploit}: Whether the agent prioritizes information gain or risk minimization
\end{itemize}

\begin{figure}[H]
\centering
\includegraphics[width=0.75\textwidth]{figures/fig1_four_quadrant.pdf}
\caption{The Four Motivational Quadra derived from crossing Self/Other orientation with Explore/Exploit behavioral mode. Each quadrant represents a distinct free-energy minimization target: \textit{Seeking} (Self $\times$ Explore) involves fantasy development and internal elaboration; \textit{Conferring} (Other $\times$ Explore) involves surveillance and information gathering; \textit{Revising} (Self $\times$ Exploit) involves ritualization and pattern consolidation; \textit{Directing} (Other $\times$ Exploit) involves action, control, and behavioral execution.}
\label{fig:quadra}
\end{figure}

\begin{table}[H]
\centering
\caption{The four motivational states and their free-energy targets.}
\label{tab:quadra}
\small
\begin{tabular}{llp{3.5cm}p{4cm}}
\toprule
\textbf{State} & \textbf{Axes} & \textbf{FE Target} & \textbf{Behavioral Signature} \\
\midrule
Seeking & Self $\times$ Explore & Maximize epistemic gain for self & Fantasy, introspection, curiosity \\
Directing & Other $\times$ Exploit & Minimize risk for other & Control, manipulation, violence \\
Conferring & Other $\times$ Explore & Maximize epistemic gain for other & Observation, stalking, learning \\
Revising & Self $\times$ Exploit & Minimize risk for self & Rituals, compulsions, consolidation \\
\bottomrule
\end{tabular}
\end{table}

\subsection{Mathematical Properties of the Four-State Space}

The four-state space is designed to satisfy several desirable properties \citep{senthil2026comppsy}:

\begin{enumerate}
    \item \textbf{Mutual exclusivity}: Each observation maps to exactly one state (enforced by classification procedure)
    \item \textbf{Collective exhaustiveness}: Every possible observation is representable (the 2$\times$2 crossing covers all combinations)
    \item \textbf{Balanced structure}: The symmetric crossing of dimensions encourages approximately uniform priors
    \item \textbf{Minimal category count}: Four states is the smallest number that captures both dimensions
\end{enumerate}

\textbf{Note}: The original Computational Psychodynamics framework \citep{senthil2026comppsy} derives these properties from the Steiner system $S(3,4,8)$. We do not reproduce that derivation here; interested readers should consult the framework paper. For present purposes, the key point is that the 2$\times$2 structure provides a principled, minimal state space for Markov analysis.

\subsection{Markov Chain Dynamics}

Each agent's behavioral stream is modeled as a time-inhomogeneous four-state Markov chain with softmax action selection:

\begin{equation}
    P(Z_{t+1} = j \mid Z_t = i) = \frac{\exp\left[Q_t(i, j) / T_t\right]}{\sum_k \exp\left[Q_t(i, k) / T_t\right]}
\end{equation}

where $Q_t(i, j)$ is the expected value of transitioning from state $i$ to state $j$ (derived from free-energy considerations), and $T_t > 0$ is a temperature parameter controlling exploration. Higher $Q$ values yield higher transition probabilities. The resulting $4 \times 4$ transition matrix $\mathbf{K}_t$ encodes an individual's characteristic motivational dynamics---their behavioral ``fingerprint.''

\subsection{Transfer Entropy for Detecting Shared Structure}

For two behavioral sequences $X$ and $Y$, transfer entropy quantifies directed predictive information:

\begin{equation}
    TE_{X \rightarrow Y} = H(Y_t \mid Y_{t-1}) - H(Y_t \mid Y_{t-1}, X_{t-1})
\end{equation}

High $TE(A \rightarrow B)$ indicates that individual $A$'s behavioral pattern provides predictive information about individual $B$'s trajectory---even when these individuals never met and lived in different eras. This cross-lifetime predictive influence is the mathematical basis for \textit{archetypal reincarnation}: the same generative patterns manifest repeatedly across different lives, detectable as non-zero transfer entropy between trajectories separated by space and time.

\subsection{Adaptation to Criminal Trajectory Analysis}

The original Computational Psychodynamics framework was validated on conversational turn-taking (seconds-to-minutes timescale). Criminal life histories operate at radically different scales (years-to-decades) with sparse, retrospective data. Key adaptations include:

\begin{enumerate}
    \item \textbf{Event-level rather than utterance-level classification}: Each documented life event (rather than conversational turn) is classified into the four-state space
    \item \textbf{LLM-assisted classification}: Using large language models with chain-of-thought prompting for robust state assignment
    \item \textbf{Lexical imputation}: Generating paraphrases to handle variation in how events are described across sources
    \item \textbf{Phase-normalized comparison}: Aligning sequences by life phase rather than absolute time for transfer entropy computation
\end{enumerate}

%==============================================================================
\section{Study 1: Behavioral Classification and Markov Analysis}
%==============================================================================

\subsection{Data Source and Sample}

Data were drawn from the Radford University Serial Killer Database, supplemented by published case materials. Inclusion criteria required $\geq 20$ documented life events with sufficient detail for state classification.

\begin{table}[H]
\centering
\caption{Sample characteristics ($N = 26$).}
\begin{tabular}{ll}
\toprule
\textbf{Characteristic} & \textbf{Value} \\
\midrule
Total events classified & 1,246 \\
Events per individual & $M = 47.9$, $SD = 28.3$, range = 21--134 \\
Sex: Male & 24 (92.3\%) \\
Sex: Female & 2 (7.7\%) \\
Confirmed victims & $M = 10.8$, $SD = 7.9$ \\
Active years & $M = 8.4$, $SD = 6.2$ \\
\bottomrule
\end{tabular}
\end{table}

\subsection{Classification Pipeline}

\subsubsection{Lexical Imputation}

To address variation in how the same event is described across sources, we generated five paraphrases per event using GPT-4o-mini (temperature = 0.7). The centroid embedding of original plus paraphrases provides robust, lexically-invariant representations.

\subsubsection{Semantic Embedding}

Event descriptions were embedded using sentence-transformers (all-MiniLM-L6-v2), yielding 384-dimensional vectors.

\subsubsection{State Classification}

GPT-4o with chain-of-thought prompting classified each event into the four-state space. The prompt included:
\begin{itemize}
    \item Definitions of all four states with criminal-specific examples
    \item The event description
    \item Instructions to provide reasoning before classification
    \item Confidence score (0--1)
\end{itemize}

Validation against two trained human raters on 100 events yielded $\kappa = 0.76$ (substantial agreement), comparable to inter-human agreement ($\kappa = 0.78$).

\subsection{Results: State Distribution}

\begin{table}[H]
\centering
\caption{Aggregate state distribution across all events.}
\begin{tabular}{lrrr}
\toprule
\textbf{State} & \textbf{Count} & \textbf{Percentage} & \textbf{95\% CI} \\
\midrule
Directing & 476 & 38.2\% & [35.5, 40.9] \\
Seeking & 300 & 24.1\% & [21.7, 26.5] \\
Conferring & 247 & 19.8\% & [17.6, 22.0] \\
Revising & 223 & 17.9\% & [15.8, 20.0] \\
\bottomrule
\end{tabular}
\end{table}

\textbf{Interpretation}: Directing dominance (38.2\%) is expected given that documented events in serial offender histories disproportionately capture offense-related behaviors. The substantial Seeking (24.1\%) and Conferring (19.8\%) components capture fantasy development and victim surveillance phases.

\begin{figure}[H]
\centering
\includegraphics[width=0.65\textwidth]{figures/fig3_stationary_distribution.pdf}
\caption{Stationary distribution showing long-term state occupancy probabilities. The dashed line indicates the uniform distribution (0.25). Seeking and Directing show slightly elevated occupancy, reflecting their role as attractor states in the motivational dynamics.}
\label{fig:stationary}
\end{figure}

\subsection{Results: Transition Structure}

\begin{table}[H]
\centering
\caption{Key transitions with psychological interpretations.}
\begin{tabular}{llrl}
\toprule
\textbf{Transition} & \textbf{Probability} & \textbf{Count} & \textbf{Interpretation} \\
\midrule
Directing $\rightarrow$ Directing & 0.42 & 312 & Self-reinforcing violence \\
Seeking $\rightarrow$ Directing & 0.34 & 187 & Fantasy-to-action escalation \\
Conferring $\rightarrow$ Directing & 0.31 & 142 & Observation-to-action \\
Revising $\rightarrow$ Revising & 0.38 & 89 & Entrenched ritualization \\
Seeking $\rightarrow$ Seeking & 0.29 & 134 & Prolonged internal struggle \\
\bottomrule
\end{tabular}
\end{table}

\textbf{Key finding}: The Seeking$\rightarrow$Directing transition ($p = 0.34$) represents the critical ``fantasy-to-action'' pathway---the moment when internal urges translate into external violence. This transition becomes a primary target for intervention.

\begin{figure}[H]
\centering
\includegraphics[width=0.7\textwidth]{figures/fig2_transition_matrix.pdf}
\caption{Aggregate transition matrix showing state-to-state transition probabilities ($n = 26$ individuals, 1,246 events). Darker cells indicate higher transition probabilities. The Directing$\rightarrow$Directing self-loop (0.42) reflects the self-reinforcing nature of violent behavior. The Seeking$\rightarrow$Directing transition (0.34) represents the critical ``fantasy-to-action'' escalation pathway.}
\label{fig:transition}
\end{figure}

\subsection{Results: Individual Variation}

Substantial heterogeneity exists across individuals:
\begin{itemize}
    \item Directing proportion ranges from 22\% to 78\%
    \item Some individuals show escalation (increasing Directing over career); others remain stable
    \item Entropy (behavioral complexity) ranges from 1.12 to 1.96 bits
\end{itemize}

\begin{figure}[H]
\centering
\includegraphics[width=0.95\textwidth]{figures/fig5_example_trajectory.pdf}
\caption{Example behavioral trajectory showing state transitions over chronologically ordered life events. The red arrow indicates a critical Seeking$\rightarrow$Directing transition---the ``fantasy-to-action'' escalation that represents a primary intervention target. Note the oscillation between Seeking and Conferring states before escalation, characteristic of pre-offense behavioral patterns.}
\label{fig:trajectory}
\end{figure}

This variation motivates the hierarchical classification system developed in Study 3.

%==============================================================================
\section{Study 2: Archetypal Discovery via Transfer Entropy}
%==============================================================================

\subsection{Rationale: Recurring Behavioral Patterns Across Individuals}

Examining case histories, we observed a striking pattern: similar behavioral sequences appear across individuals who never met, were separated by decades, and could not have directly influenced each other. We term this phenomenon \textit{archetypal recurrence} (or, metaphorically, ``archetypal reincarnation'')---the statistical similarity of behavioral patterns across unconnected individuals.

\textbf{Terminology note}: We use ``reincarnation'' as a \textit{metaphor} for the mathematical phenomenon of shared generative structure, not as a literal claim. Transfer entropy detects predictive relationships between sequences---when $TE(A \rightarrow B) > 0$, individual $A$'s trajectory provides information about individual $B$'s trajectory. This could reflect shared environmental pressures, common developmental pathways, or similar psychological dynamics---not supernatural transmission.

\subsubsection{Theoretical Foundation: Attachment as Influence Across Time}

This phenomenon has deep roots in attachment theory \citep{bowlby1969attachment, ainsworth1978patterns}. The core mechanism of attachment is \textit{influence over another's future behavior}: internal working models (IWMs) formed through early relationships shape how individuals act, relate, and regulate throughout life. Critically, attachment patterns transmit intergenerationally---a parent's attachment style influences their parenting, which shapes the child's attachment, which shapes their adult relationships and parenting in turn \citep{main1985security}.

Transfer entropy formalizes this mechanism mathematically. When $TE(A \rightarrow B) > 0$, individual $A$'s behavioral sequence reduces uncertainty about individual $B$'s future states:

\begin{equation}
    TE(A \rightarrow B) = H(B_t \mid B_{t-1}) - H(B_t \mid B_{t-1}, A_{t-1})
\end{equation}

This is precisely what attachment accomplishes: one person's experiential structure providing predictive information about another's trajectory. The attachment figure's pattern ($A_{t-1}$) adds information beyond what the individual's own history ($B_{t-1}$) provides---the mathematical signature of relational influence.

\subsubsection{Beyond Linear Time: Archetypes as Attachment Templates}

Traditional attachment research assumes linear temporal causation: parent influences child through direct interaction. But if we relax this assumption, a deeper pattern emerges. The individuals in our dataset never met---they were separated by decades, geography, and social context. Yet their behavioral trajectories predict each other.

This is not paradoxical; it is the logical extension of attachment transmission. Attachment patterns do not require direct contact to propagate---they are \textit{generative templates} that manifest wherever the conditions for their expression exist. A ``disorganized attachment'' pattern can emerge in individuals across different eras because the template itself is latent in human relational possibility space.

\textbf{Archetypes}, in this framework, are these latent attachment templates---the underlying generative structures that produce similar trajectories across individuals. When multiple unconnected individuals exhibit high mutual transfer entropy, they share an underlying archetypal structure: the same relational template manifesting in different lives.

The term \textit{reincarnation} captures this metaphorically. These templates do not die with individuals---they persist as patterns that can be re-instantiated wherever similar conditions exist. When we detect $TE(A \rightarrow B) > 0$ between individuals who never interacted, we are detecting shared generative structure: \textit{as if} the same experiential template were influencing behavior across lifetimes. This is not mysticism but statistics: high mutual transfer entropy indicates that unrelated individuals exhibit predictably similar behavioral dynamics, likely due to common underlying psychological processes rather than direct influence.

\textbf{Alternative interpretation}: The observed pattern similarity could also reflect (1) homogeneity in our sample (all serial offenders from similar cultural contexts), (2) common data sources (Radford database) introducing correlated measurement, or (3) genuine convergent evolution of criminal behavioral strategies. We cannot distinguish these explanations with the current data.

\subsubsection{The Four States as Attachment Dynamics}

The Computational Psychodynamics states map onto attachment processes:

\begin{itemize}
    \item \textbf{Seeking}: Exploration from secure base, or anxious proximity-seeking when the base is unavailable; fantasy as substitute attachment
    \item \textbf{Directing}: Controlling attachment behavior---domination as a strategy when secure attachment fails
    \item \textbf{Conferring}: Hypervigilant monitoring of attachment figures; surveillance as pathological attachment-seeking
    \item \textbf{Revising}: Consolidating internal working models; ritualized self-regulation when external co-regulation is unavailable
\end{itemize}

The critical Seeking$\rightarrow$Directing transition can be understood as attachment system failure: when internal regulation through fantasy (Seeking) fails to achieve equilibrium, the individual escalates to external control (Directing). Violence emerges as a pathological attempt to regulate what cannot be securely attached to---consistent with the trajectory from disorganized attachment to controlling behavior in adulthood \citep{main1990procedures}.

Computational Psychodynamics provides the mathematical tools to operationalize these concepts: \textbf{transfer entropy} measures directed influence between life trajectories (the mechanism of attachment transmission), while \textbf{network analysis} reveals the hierarchical structure of archetypal relationships (the topology of how templates propagate).

\subsection{Methods}

\subsubsection{Pairwise Transfer Entropy}

For all 26 $\times$ 25 = 650 ordered pairs of individuals, we computed:

\begin{equation}
    TE(X \rightarrow Y) = \sum_{y_{t+1}, y_t, x_t} p(y_{t+1}, y_t, x_t) \log_2 \frac{p(y_{t+1} \mid y_t, x_t)}{p(y_{t+1} \mid y_t)}
\end{equation}

where the sum is over all state combinations. This measures the information (in bits) that $X$'s past provides about $Y$'s future, beyond what $Y$'s own past provides.

\textbf{Phase normalization procedure}: Because individuals have different numbers of documented events (range: 21--134), we resampled all sequences to a common length of 50 time points using linear interpolation of state indices. This aligns ``life phases'' (e.g., early career, peak offending, late career) rather than absolute chronological time. We acknowledge this introduces interpolation artifacts; sensitivity analysis with common lengths of 30, 50, and 70 showed qualitatively similar network structure (Jaccard similarity of thresholded edges $> 0.75$).

\subsubsection{Network Construction}

The $26 \times 26$ TE matrix was thresholded at the 85th percentile of non-zero values to construct a directed graph where nodes are individuals and edges represent high predictive relationships.

\subsubsection{Role Assignment}

Network roles were assigned based on incoming and outgoing TE:
\begin{itemize}
    \item \textbf{Source}: Outgoing $> \mu + 1.5\sigma$, Incoming $< \mu + 0.5\sigma$
    \item \textbf{Sink}: Incoming $> \mu + 1.5\sigma$, Outgoing $< \mu + 0.5\sigma$
    \item \textbf{Hub}: Both $> \mu + \sigma$
\end{itemize}

\subsection{Results: Network Structure}

\begin{table}[H]
\centering
\caption{Transfer entropy network statistics.}
\begin{tabular}{ll}
\toprule
\textbf{Metric} & \textbf{Value} \\
\midrule
Mean TE (non-zero) & 0.23 bits ($SD = 0.18$) \\
Network density (at threshold) & 0.134 \\
Permutation test $p$-value & $< 0.001$ \\
\bottomrule
\end{tabular}
\end{table}

The permutation test confirms that the network structure is significantly non-random---behavioral sequences contain genuine shared patterns not attributable to state frequency alone.

\begin{figure}[H]
\centering
\includegraphics[width=0.85\textwidth]{figures/fig4_te_network.pdf}
\caption{Transfer entropy network showing archetypal relationships between individuals. Nodes represent individuals; edges indicate significant predictive relationships (TE above 85th percentile threshold). \textbf{Sources} (red, n=3) are archetypal exemplars whose patterns predict others. \textbf{Sinks} (teal, n=3) are composite cases predicted by multiple archetypes. \textbf{Hubs} (yellow, n=4) bridge different archetypal clusters. The network structure quantifies how behavioral patterns propagate across lifetimes via shared generative structure.}
\label{fig:network}
\end{figure}

\subsection{Results: Archetypal Roles}

\begin{table}[H]
\centering
\caption{Network role distribution.}
\begin{tabular}{lrrr}
\toprule
\textbf{Role} & \textbf{n} & \textbf{\%} & \textbf{Interpretation} \\
\midrule
Source & 3 & 11.5\% & Archetypal exemplars \\
Sink & 3 & 11.5\% & Composite cases \\
Hub & 4 & 15.4\% & Central connectors \\
General & 16 & 61.5\% & --- \\
\bottomrule
\end{tabular}
\end{table}

\begin{keyinsight}
\textbf{Sources} exhibit prototypical patterns that predict many others but are themselves unpredicted---they are the ``archetypal templates,'' the purest instantiations of a generative pattern. \textbf{Sinks} are predicted by many sources, suggesting their life trajectories combine elements from multiple archetypes---composite patterns drawing on several experiential templates. \textbf{Hubs} bridge different archetypal clusters, their trajectories carrying information that connects otherwise distinct patterns.
\end{keyinsight}

These roles have concrete interpretations in terms of experiential influence. A Source individual's life trajectory contains information that reduces uncertainty about how other lives will unfold---their pattern ``propagates'' across lifetimes. A Sink individual's trajectory is predictable from multiple prior patterns---their life represents a ``convergence'' of archetypal influences. This network structure quantifies what clinicians have long intuited: that certain offenders are ``textbook cases'' (Sources) while others represent complex mixtures (Sinks).

\subsection{Results: Archetypal Lineages}

We extracted 20 ``lineages''---chains of sequential high-TE relationships representing coherent archetypal threads through the network. The longest lineage (6 individuals) traces a gradient from fantasy-driven to action-dominant patterns.

%==============================================================================
\section{Study 3: Hierarchical Classification System}
%==============================================================================

\subsection{Rationale}

Network roles are continuous; clinical practice requires discrete types. We developed a two-level hierarchical system:
\begin{enumerate}
    \item \textbf{Level 1 (Data-driven)}: Primary types emerging from clustering
    \item \textbf{Level 2 (Theory-driven)}: Subtypes within each primary type based on Computational Psychodynamics principles
\end{enumerate}

\subsection{Methods}

\subsubsection{Level 1: Primary Types}

A 9-dimensional feature vector was extracted for each individual:
\begin{itemize}
    \item State distribution (4 features): Proportion in each state
    \item State persistence (4 features): Self-loop probability for each state
    \item Escalation (1 feature): Change in Directing proportion from early to late career
\end{itemize}

Ward's hierarchical clustering with silhouette analysis identified $k = 2$ as optimal, yielding:
\begin{itemize}
    \item \textbf{COMPLEX}: Lower Directing, higher entropy, multi-modal state distribution
    \item \textbf{FOCUSED}: Higher Directing, lower entropy, state-dominant
\end{itemize}

\subsubsection{Level 2: Subtypes}

Within each primary type, theory-driven criteria assigned subtypes:

\begin{table}[H]
\centering
\caption{Subtype definitions.}
\small
\begin{tabular}{llp{4cm}p{3.5cm}}
\toprule
\textbf{Primary} & \textbf{Subtype} & \textbf{Criteria} & \textbf{Psychology} \\
\midrule
COMPLEX & Chameleon & $\geq 3$ active states, $<60\%$ any & Highly adaptive \\
COMPLEX & Multi-Modal & 2+ states $>25\%$ & Variable patterns \\
\midrule
FOCUSED & Pure Predator & Directing $\geq 75\%$ & Sustained exploitation \\
FOCUSED & Strong Escalator & Escalation $\geq 0.35$ & Clear trajectory increase \\
FOCUSED & Stalker-Striker & Conf$\rightarrow$Dir present & Methodical \\
FOCUSED & Fantasy-Actor & Seek$\rightarrow$Dir, no Conf$\rightarrow$Dir & Impulsive \\
FOCUSED & Standard & Default FOCUSED & Typical pattern \\
\bottomrule
\end{tabular}
\end{table}

\subsection{Results}

\begin{table}[H]
\centering
\caption{Hierarchical classification distribution ($N = 26$).}
\begin{tabular}{llrr}
\toprule
\textbf{Primary} & \textbf{Subtype} & \textbf{n} & \textbf{\%} \\
\midrule
COMPLEX & & 3 & 11.5\% \\
 & Chameleon & 1 & 3.8\% \\
 & Multi-Modal & 2 & 7.7\% \\
\midrule
FOCUSED & & 23 & 88.5\% \\
 & Standard & 15 & 57.7\% \\
 & Pure Predator & 3 & 11.5\% \\
 & Strong Escalator & 2 & 7.7\% \\
 & Fantasy-Actor & 2 & 7.7\% \\
 & Stalker-Striker & 1 & 3.8\% \\
\bottomrule
\end{tabular}
\end{table}

\textbf{Validation}: Split-half reliability was $\kappa = 0.83$ for primary type and $\kappa = 0.71$ for subtype. Expert validation (three forensic psychologists) rated 82\% of classifications as ``accurate'' or ``very accurate.''

\begin{figure}[H]
\centering
\includegraphics[width=0.9\textwidth]{figures/fig6_hierarchical_tree.pdf}
\caption{Hierarchical classification system showing the two-level typology. \textbf{Level 1} distinguishes COMPLEX (11.5\%) from FOCUSED (88.5\%) types based on entropy and state distribution. \textbf{Level 2} assigns theory-driven subtypes within each primary type. Classification criteria are shown in the lower left. This system preserves individual variation while providing clinically actionable categories.}
\label{fig:hierarchy}
\end{figure}

%==============================================================================
\section{Study 4: Causal Modeling and Intervention Reasoning}
%==============================================================================

\subsection{From Description to Prescription: An Illustrative Framework}

Studies 1--3 characterized patterns; Study 4 develops a \textit{theoretical framework} for intervention reasoning. We construct a Structural Causal Model (SCM) that supports counterfactual analysis via do-calculus \citep{pearl2009causality}.

\textbf{Important caveat}: The intervention estimates in this section are \textit{model-based projections} derived from assumed effect sizes drawn from the treatment literature, not empirical treatment effects measured in this sample. They illustrate how the framework could guide intervention reasoning, pending prospective validation.

\subsection{Structural Causal Model}

\subsubsection{Graph Structure}

\begin{itemize}
    \item \textbf{Nodes}: Behavioral states at each time point
    \item \textbf{Edges}: Transition probabilities from Markov analysis
    \item \textbf{Intervention nodes}: External manipulations that modify transition probabilities
    \item \textbf{Outcome nodes}: Terminal states (e.g., reached\_Directing, harm\_occurred)
\end{itemize}

\subsubsection{Do-Calculus Implementation}

The do-operator $do(X = x)$ models intervention by:
\begin{enumerate}
    \item Removing incoming edges to $X$ (breaking confounding)
    \item Setting $X = x$ deterministically
    \item Computing downstream effects
\end{enumerate}

This allows causal queries: ``What is $P(\text{harm} \mid do(\text{intervention at } t))$?''

\subsection{Intervention Protocol Library}

We compiled 14 evidence-based protocols grounded in RNR principles \citep{andrews2010RNR} and therapeutic frameworks. Effect sizes are drawn from meta-analyses of treatment outcomes in forensic populations; their application to serial offenders specifically remains untested:

\begin{table}[H]
\centering
\caption{Selected intervention protocols.}
\small
\begin{tabular}{llll}
\toprule
\textbf{Protocol} & \textbf{Target State} & \textbf{Mechanism} & \textbf{Evidence} \\
\midrule
CBT Fantasy Management & Seeking & Cognitive restructuring & A \\
DBT Impulse Control & Directing & Emotion regulation & A \\
Schema Therapy & Multiple & Schema modification & B \\
Intensive Supervision & Conferring & Opportunity reduction & A \\
Comprehensive Program & All & Combined approach & A \\
\bottomrule
\end{tabular}
\end{table}

Each protocol specifies:
\begin{itemize}
    \item Effect on specific transition probabilities (e.g., CBT reduces $P(\text{Seeking} \rightarrow \text{Directing})$ by 30\%)
    \item Intensity levels and duration
    \item Contraindications
    \item Theoretical basis and mechanism of action
\end{itemize}

\textbf{Source of effect sizes}: Transition probability modifications are \textit{assumed} based on meta-analytic effect sizes for general forensic populations. For example, the 30\% reduction for CBT is derived from \citet{andrews2010RNR} reporting $d \approx 0.30$ for cognitive-behavioral interventions. We translate standardized effect sizes to transition probability modifications using $\Delta P \approx d \cdot \sigma_P$, where $\sigma_P$ is the standard deviation of the baseline transition probability. This translation involves substantial uncertainty and should be considered illustrative.

\subsection{Critical Transition Analysis}

\begin{table}[H]
\centering
\caption{Critical transitions ranked by harm potential.}
\begin{tabular}{lrrl}
\toprule
\textbf{Transition} & \textbf{Frequency} & \textbf{Probability} & \textbf{Risk Score} \\
\midrule
Seeking $\rightarrow$ Directing & 187 & 0.34 & 0.34 (highest) \\
Conferring $\rightarrow$ Directing & 142 & 0.31 & 0.31 \\
Directing $\rightarrow$ Directing & 312 & 0.42 & 0.34 (weighted) \\
\bottomrule
\end{tabular}
\end{table}

\textbf{Primary intervention target}: The Seeking$\rightarrow$Directing transition represents the ``fantasy-to-action'' pathway---the critical moment when internal urges translate into external violence.

\subsection{Tipping Point Analysis}

Using absorbing Markov chain analysis, we identified \textit{tipping points}---moments where the probability of eventually reaching the Directing state exceeds 0.6:

\begin{itemize}
    \item Mean tipping point: Event 23 ($SD = 12.4$)
    \item Post-tipping intervention requires greater intensity
    \item Earlier tipping points associated with higher Seeking proportion and faster escalation
\end{itemize}

\begin{figure}[H]
\centering
\includegraphics[width=0.85\textwidth]{figures/fig7_intervention_window.pdf}
\caption{Intervention window analysis showing optimal timing for intervention. \textbf{Top panel}: Probability of reaching Directing state over time, with the tipping point (P=0.5) marked by dashed line. The green shaded region indicates the intervention window where interventions are most effective; the red shaded region indicates the high-risk zone post-tipping point. \textbf{Bottom panel}: Corresponding state trajectory showing the critical Seeking$\rightarrow$Directing transition. Earlier intervention (within the green window) yields 2--3$\times$ greater harm reduction compared to post-tipping intervention.}
\label{fig:intervention}
\end{figure}

\subsection{Counterfactual Simulation: Illustrative Projections}

Using Monte Carlo methods, we simulated counterfactual trajectories to \textit{illustrate} how the framework could estimate potential intervention effects. For each individual, we asked: ``What would have happened if we intervened at event $k$ with protocol $P$?''

\textbf{Critical limitation}: These estimates are model-based projections that depend on (1) assumed intervention effect sizes from the general forensic literature, (2) the validity of our Markov model, and (3) correct causal graph structure. They should be interpreted as \textit{theoretical illustrations} of the framework's potential, not as validated treatment effect estimates.

\subsubsection{Three-Step Process}
\begin{enumerate}
    \item \textbf{Abduction}: Infer latent parameters from observed trajectory
    \item \textbf{Action}: Apply intervention (modify transition probabilities)
    \item \textbf{Prediction}: Monte Carlo simulation ($N = 1,000$) of counterfactual trajectories
\end{enumerate}

\subsubsection{Results}

\begin{table}[H]
\centering
\caption{Projected counterfactual harm reduction by intervention timing (illustrative simulation based on assumed effect sizes from forensic treatment literature).}
\begin{tabular}{llrrl}
\toprule
\textbf{Timing} & \textbf{Best Protocol} & \textbf{Harm Reduction} & \textbf{95\% CI} & \textbf{NNT} \\
\midrule
First Seeking state & CBT Fantasy & 47.2\% & [38.1, 56.3] & 2.1 \\
First Conferring & Intensive Supervision & 34.8\% & [26.2, 43.4] & 2.9 \\
Pre-tipping point & DBT Impulse & 31.4\% & [22.8, 40.0] & 3.2 \\
Post-tipping point & Comprehensive & 18.6\% & [11.2, 26.0] & 5.4 \\
\bottomrule
\end{tabular}
\end{table}

\begin{keyinsight}
\textit{If the assumed effect sizes are accurate}, the model projects that early intervention would be 2--3$\times$ more effective than late intervention, with NNT approximately doubling post-tipping point. These projections require prospective validation before clinical application.
\end{keyinsight}

\subsection{Optimization Framework}

\subsubsection{Objective}
\begin{equation}
    \max_{\text{protocol}, t} \mathbb{E}[\text{harm reduction}] - \lambda \cdot \text{cost}
\end{equation}

\subsubsection{Results}
\begin{itemize}
    \item \textbf{Optimal single intervention}: Early Seeking-phase CBT (cost-effectiveness ratio: 0.47 harm reduction per \$10K)
    \item \textbf{Optimal sequence}: CBT $\rightarrow$ DBT $\rightarrow$ Supervision (mean 3.2 interventions)
    \item Diminishing returns after 4th intervention
\end{itemize}

%==============================================================================
\section{Study 5: State Space Validation}
%==============================================================================

\subsection{Rationale: Theory vs. Data-Driven State Spaces}

A fundamental question in behavioral sequence analysis is: \textit{How should we partition the space of events into discrete states for Markov analysis?} Two approaches exist:

\begin{enumerate}
    \item \textbf{Theory-driven}: States derived from psychological framework (e.g., the 4-Animal quadra from Computational Psychodynamics)
    \item \textbf{Data-driven}: States emerge from clustering events in embedding space
\end{enumerate}

We validate our theory-driven approach by comparing it against data-driven alternatives using information-theoretic metrics and proper null hypothesis testing.

\subsection{Methods}

\subsubsection{Data-Driven State Space}

Events were embedded using sentence-transformers and clustered via K-means. Silhouette analysis identified $K = 10$ as optimal, yielding clusters with interpretable themes (e.g., ``Predatory Stalking,'' ``Sexual Violence,'' ``Search for Identity'').

\subsubsection{Mapping Between State Spaces}

Two mappings from 10 clusters to 4 states were computed:

\begin{itemize}
    \item \textbf{Theoretical mapping}: Each cluster assigned to the 4-Animal state based on semantic/psychological alignment
    \item \textbf{Optimal mapping}: Exhaustive search over all $4^{10} = 1,048,576$ possible mappings to maximize information retention
\end{itemize}

\subsubsection{Information Retention}

Information retention quantifies structure preservation when mapping from fine-grained to coarse-grained states \citep{shannon1948mathematical, cover2006elements}:

\begin{equation}
    \text{Information Retention} = \frac{I(S; T)}{H(S)}
\end{equation}

where $I(S; T)$ is mutual information between source states $S$ and target states $T$, and $H(S)$ is entropy of the source distribution.

\subsubsection{Three Null Hypothesis Tests}

\begin{enumerate}
    \item \textbf{Mapping Null}: Is the theoretical mapping better than random mappings?
    \begin{equation}
        H_0: I(S; \phi_{\text{theory}}(S)) = I(S; \phi_{\text{random}}(S))
    \end{equation}
    Tested via 10,000 permutations of random cluster-to-state assignments.

    \item \textbf{Sequence Null}: Do observed transitions differ from shuffled sequences?
    \begin{equation}
        H_0: P_{\text{observed}} = P_{\text{shuffled}}
    \end{equation}
    Tested by computing transition matrices on shuffled sequences preserving state frequencies.

    \item \textbf{Predictive Null}: Does state-conditional prediction beat marginal prediction?
    \begin{equation}
        H_0: P(s_{t+1} \mid s_t) = P(s_{t+1})
    \end{equation}
    Tested by comparing predictive accuracy.
\end{enumerate}

\subsection{Results}

\subsubsection{Information Retention}

\begin{table}[H]
\centering
\caption{Information retention by mapping approach.}
\begin{tabular}{lrr}
\toprule
\textbf{Mapping} & \textbf{Retention} & \textbf{MI (bits)} \\
\midrule
Exhaustive Optimal & 72.4\% & 1.978 \\
Greedy Optimal & 72.4\% & 1.977 \\
Theoretical & 65.0\% & 1.776 \\
Spectral & 52.7\% & 1.439 \\
\bottomrule
\end{tabular}
\end{table}

The theoretical mapping retains 65.0\% of information---7.4 percentage points below the optimal. However, the optimal mapping differs from theoretical on 7 of 10 clusters.

\subsubsection{Null Test Results}

\begin{table}[H]
\centering
\caption{State space validation: Three null hypothesis tests.}
\small
\begin{tabular}{lp{2.5cm}rrl}
\toprule
\textbf{Test} & \textbf{Statistic} & \textbf{Effect Size} & \textbf{$p$-value} & \textbf{Result} \\
\midrule
1. Mapping Null & Retention = 0.65 & $d = 0.87$ & 0.21 & Not significant \\
2. Sequence Null & $\chi^2(9) = 142.3$ & $d = 12.5$ & $< .0001$ & Significant \\
3. Predictive Null & $\Delta$Acc = +12.6\% & $d = 4.2$ & $< .0001$ & Significant \\
\bottomrule
\end{tabular}
\end{table}

\textit{Note.} Test 1: Theoretical mapping vs. random mappings (10,000 permutations). Test 2: Observed vs. shuffled transition matrices, $df = (4-1)^2 = 9$ for a $4 \times 4$ matrix. Test 3: Markov prediction vs. marginal baseline.

\subsubsection{Key Finding}

\begin{keyinsight}
The 4-state representation captures \textbf{highly significant temporal structure} in behavioral sequences (Tests 2--3, $p < 0.0001$). However, the specific theoretical cluster-to-state mapping is \textbf{not statistically superior to random mappings} (Test 1, $p = 0.21$). This is a \textit{negative result} for the theoretical mapping specifically, though not for the framework overall.
\end{keyinsight}

\textbf{Interpretation}: The non-significant mapping test ($p = 0.21$) means we cannot claim the Computational Psychodynamics categories are statistically ``correct''---many alternative 4-state partitions capture similar amounts of information. The framework's value lies in its \textit{interpretability} (states have psychological meaning) and \textit{theoretical coherence} (grounded in Free Energy Principle), not statistical optimality. We accept 7.4\% information loss relative to the optimal mapping in exchange for categories that clinicians and researchers can reason about.

\subsubsection{Mapping Comparison}

The optimal mapping differs from theoretical on clusters associated with:
\begin{itemize}
    \item Predatory stalking (Theoretical: Conferring $\rightarrow$ Optimal: Seeking)
    \item Sexual violence (Theoretical: Directing $\rightarrow$ Optimal: Seeking)
    \item Domestic conflict (Theoretical: Conferring $\rightarrow$ Optimal: Directing)
\end{itemize}

These differences suggest that while the theoretical categories have psychological coherence, the statistical structure of the data groups events differently. The optimal mapping produces more balanced state distributions (22--33\% per state) compared to the theoretical mapping (10--40\% per state).

\subsection{Discussion}

The validation results support a nuanced interpretation:

\begin{enumerate}
    \item \textbf{The 4-state dimensionality is appropriate}: Significant temporal structure exists and is captured ($p < 0.0001$)
    \item \textbf{Markov modeling is justified}: Conditioning on previous state substantially improves prediction (+12.6\%)
    \item \textbf{The specific mapping is not statistically optimal}: The theoretical mapping ($p = 0.21$) is chosen for interpretability, not statistical superiority
    \item \textbf{Interpretability vs. optimality trade-off}: We accept 7.4\% information loss for psychological coherence
\end{enumerate}

This hybrid approach---using theory-driven states validated against data-driven structure---exemplifies the methodological philosophy of Computational Psychodynamics: ground categories in psychological theory while empirically validating their utility.

%==============================================================================
\section{General Discussion}
%==============================================================================

\subsection{Summary of Contributions}

This work demonstrates that Computational Psychodynamics \citep{senthil2026comppsy}---originally developed for conversational dynamics---generalizes to criminal trajectory analysis, providing:

\begin{enumerate}
    \item \textbf{Behavioral fingerprints}: The four-state Markov chain captures individual differences in criminal career dynamics ($N = 26$; Study 1)
    \item \textbf{Pattern similarity}: Transfer entropy detects shared behavioral patterns across unconnected individuals, suggesting common generative dynamics (Study 2)
    \item \textbf{Hierarchical classification}: COMPLEX/FOCUSED + 7 subtypes provides a descriptive taxonomy (exploratory, $N = 26$; Study 3)
    \item \textbf{Intervention reasoning}: SCM + do-calculus provides a \textit{theoretical framework} for counterfactual analysis (illustrative; Study 4)
    \item \textbf{Partial validation}: The 4-state dimensionality captures temporal structure ($p < 0.0001$), though the specific mapping is not statistically superior to alternatives ($p = 0.21$; Study 5)
\end{enumerate}

\subsection{Theoretical Integration}

\subsubsection{Criminal Career Paradigm}

The Computational Psychodynamics states map onto career constructs:
\begin{itemize}
    \item \textbf{Seeking}: Onset/development phase
    \item \textbf{Conferring}: Target selection/specialization
    \item \textbf{Directing}: Active offending/persistence
    \item \textbf{Revising}: MO consolidation/habituation
\end{itemize}

Escalation score directly measures trajectory change---a core career parameter.

\subsubsection{Life-Course Criminology}

\begin{itemize}
    \item \textbf{Tipping points} correspond to failed ``turning points'' \citep{sampson1993crime}
    \item \textbf{Intervention windows} represent opportunities to manufacture turning points
    \item \textbf{Phase analysis} reveals age-graded patterns within the Markov structure
\end{itemize}

\subsubsection{Active Inference}

The four states have principled free-energy interpretations:
\begin{itemize}
    \item \textbf{Seeking}: Maximizing epistemic gain for self (fantasy as ``model-building'')
    \item \textbf{Directing}: Minimizing risk for other (violence as ``control'')
    \item \textbf{Conferring}: Maximizing epistemic gain for other (stalking as ``information gathering'')
    \item \textbf{Revising}: Minimizing risk for self (ritualization as ``consolidation'')
\end{itemize}

\subsection{Clinical Implications}

\subsubsection{Decision Support, Not Decision Making}

The framework provides information; clinical judgment remains essential. We explicitly communicate:
\begin{itemize}
    \item Confidence intervals on all estimates
    \item Alternative scenarios via counterfactual simulation
    \item Limitations and assumptions
\end{itemize}

\subsubsection{Use Cases}
\begin{itemize}
    \item \textbf{Threat assessment teams}: Identify high-risk individuals, prioritize monitoring
    \item \textbf{Forensic clinicians}: Treatment planning, protocol selection
    \item \textbf{Probation/parole}: Identify intervention windows
    \item \textbf{Research}: Hypothesis generation, pattern discovery
\end{itemize}

\subsection{Ethical Considerations}

\begin{itemize}
    \item \textbf{Pre-crime intervention}: Framework intended for treatment/supervision contexts, not prediction of future offenders
    \item \textbf{Human oversight}: All recommendations require professional judgment
    \item \textbf{Transparency}: Every metric traces to specific observable events
    \item \textbf{Privacy}: Individual-level data highly sensitive; security essential
\end{itemize}

\subsection{Limitations}

\textbf{Sample size ($N = 26$) is the primary limitation.} This affects multiple analyses:
\begin{itemize}
    \item \textbf{Transfer entropy network}: With only 26 nodes, network statistics (density, role assignments) have high variance. The 650 pairwise comparisons are not independent, inflating Type I error risk.
    \item \textbf{Hierarchical classification}: Seven subtypes from 26 individuals means some subtypes have $n = 1$--$2$. This taxonomy is \textit{descriptive}, not inferential---replication with larger samples is essential.
    \item \textbf{Transition matrix estimation}: Some cells in the $4 \times 4$ matrix have sparse counts, increasing uncertainty in probability estimates.
    \item \textbf{Generalizability}: All participants are serial offenders from similar cultural contexts (predominantly U.S.), limiting external validity.
\end{itemize}

\textbf{Additional limitations}:
\begin{itemize}
    \item \textbf{Retrospective data}: Selection bias toward well-documented, ``famous'' cases. Event documentation varies dramatically across sources.
    \item \textbf{LLM classification}: GPT-4o classification ($\kappa = 0.76$ vs. humans) may introduce systematic biases. The model was prompted with researcher-defined examples, potentially learning our categorization scheme rather than discovering structure.
    \item \textbf{Intervention effects}: Effect sizes are from general forensic literature, not this population. Translation to serial offenders is untested.
    \item \textbf{Causal assumptions}: The SCM structure is assumed, not empirically derived. Do-calculus validity depends on correct graph specification.
    \item \textbf{Data source homogeneity}: Most cases from Radford database may share documentation biases, artificially inflating pattern similarity.
\end{itemize}

\subsection{Future Directions}

\begin{enumerate}
    \item \textbf{Prospective validation}: Apply framework to ongoing cases
    \item \textbf{Expanded populations}: General offenders, domestic violence, terrorism
    \item \textbf{Real-world effect estimation}: Partner with treatment programs
    \item \textbf{Dynamic updating}: Real-time risk assessment as events occur
    \item \textbf{Closed-loop intervention}: Integration with the adaptive intervention systems described in \citet{senthil2026comppsy}
\end{enumerate}

%==============================================================================
\section{Conclusion}
%==============================================================================

We have demonstrated that Computational Psychodynamics provides a principled, mathematically grounded framework for analyzing criminal behavioral trajectories. The same four motivational states that capture moment-to-moment conversational dynamics also capture the strategic rhythm of criminal careers---from fantasy through surveillance to violence and ritualization.

Beyond description, the framework provides a \textit{theoretical basis} for intervention reasoning: modeling when interventions might have maximum leverage, which protocols could disrupt harmful transitions, and what outcomes alternative histories might have produced. This represents a potential shift from ``who is dangerous'' to ``what might we do''---from pure risk assessment toward intervention planning, though empirical validation is required before clinical application.

Every trajectory represents a life---both the offender's and potential victims'. Our goal is to identify moments where different outcomes are possible and provide guidance for creating such moments. Computational Psychodynamics, grounded in the Free Energy Principle and validated across domains, offers a path toward that goal.

%==============================================================================
% Declarations
%==============================================================================

\section*{Data Availability Statement}

The data analyzed in this study were derived from publicly available sources, including the Radford University Serial Killer Database and published case materials. Processed datasets and analysis code are available from the corresponding author upon reasonable request.

\section*{Ethics Statement}

This study analyzed retrospective, publicly available archival data from documented criminal cases. No human subjects were directly involved in data collection. The research was conducted in accordance with institutional guidelines for secondary data analysis.

\section*{Author Contributions}

\textbf{K.L.H.}: Conceptualization, data curation, validation, writing---review \& editing. \textbf{A.K.S.}: Methodology, software, formal analysis, visualization, writing---original draft.

\section*{Funding}

This research received no specific grant from any funding agency in the public, commercial, or not-for-profit sectors.

\section*{Conflicts of Interest}

The authors declare no conflicts of interest.

\section*{Acknowledgments}

The authors thank the Radford University Serial Killer Information Center for maintaining the database that made this research possible.

%==============================================================================
% References
%==============================================================================

\bibliographystyle{apalike}
\begin{thebibliography}{99}

\bibitem[Andrews \& Bonta(2010)]{andrews2010RNR}
Andrews, D. A., \& Bonta, J. (2010). \textit{The psychology of criminal conduct} (5th ed.). Anderson.

\bibitem[Canter et al.(2004)]{canter2004organized}
Canter, D., Alison, L. J., Alison, E., \& Wentink, N. (2004). The organized/disorganized typology of serial murder: Myth or model? \textit{Psychology, Public Policy, and Law}, 10(3), 293--320.

\bibitem[Douglas et al.(1986)]{douglas1986criminal}
Douglas, J. E., Ressler, R. K., Burgess, A. W., \& Hartman, C. R. (1986). Criminal profiling from crime scene analysis. \textit{Behavioral Sciences \& the Law}, 4(4), 401--421.

\bibitem[Friston(2010)]{friston2010free}
Friston, K. (2010). The free-energy principle: A unified brain theory? \textit{Nature Reviews Neuroscience}, 11, 127--138.

\bibitem[Parr et al.(2022)]{parr2022active}
Parr, T., Pezzulo, G., \& Friston, K. J. (2022). \textit{Active inference: The free energy principle in mind, brain, and behavior}. MIT Press.

\bibitem[Pearl(2009)]{pearl2009causality}
Pearl, J. (2009). \textit{Causality: Models, reasoning, and inference} (2nd ed.). Cambridge University Press.

\bibitem[Sampson \& Laub(1993)]{sampson1993crime}
Sampson, R. J., \& Laub, J. H. (1993). \textit{Crime in the making: Pathways and turning points through life}. Harvard University Press.

\bibitem[Senthil(2026)]{senthil2026comppsy}
Senthil, A. K. (2026). Computational psychodynamics: An ecological approach to computational behavioral modeling. Manuscript under review. (Full framework specification available from author upon request.)

\bibitem[Shannon(1948)]{shannon1948mathematical}
Shannon, C. E. (1948). A mathematical theory of communication. \textit{Bell System Technical Journal}, 27(3), 379--423.

\bibitem[Cover \& Thomas(2006)]{cover2006elements}
Cover, T. M., \& Thomas, J. A. (2006). \textit{Elements of information theory} (2nd ed.). Wiley-Interscience.

\bibitem[Bowlby(1969)]{bowlby1969attachment}
Bowlby, J. (1969). \textit{Attachment and loss: Vol. 1. Attachment}. Basic Books.

\bibitem[Ainsworth et al.(1978)]{ainsworth1978patterns}
Ainsworth, M. D. S., Blehar, M. C., Waters, E., \& Wall, S. N. (1978). \textit{Patterns of attachment: A psychological study of the strange situation}. Lawrence Erlbaum Associates.

\bibitem[Main et al.(1985)]{main1985security}
Main, M., Kaplan, N., \& Cassidy, J. (1985). Security in infancy, childhood, and adulthood: A move to the level of representation. \textit{Monographs of the Society for Research in Child Development}, 50(1-2), 66--104.

\bibitem[Main \& Hesse(1990)]{main1990procedures}
Main, M., \& Hesse, E. (1990). Parents' unresolved traumatic experiences are related to infant disorganized attachment status: Is frightened and/or frightening parental behavior the linking mechanism? In M. T. Greenberg, D. Cicchetti, \& E. M. Cummings (Eds.), \textit{Attachment in the preschool years} (pp. 161--182). University of Chicago Press.

\end{thebibliography}

\end{document}
